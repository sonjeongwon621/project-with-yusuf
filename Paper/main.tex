\documentclass[]{article}
\usepackage[utf8]{inputenc}
\usepackage[english]{babel}
\usepackage{graphicx}
\usepackage{hyperref}
\usepackage{appendix}
\hypersetup{
	colorlinks=true,
	linkcolor=blue,
	filecolor=magenta,      
	urlcolor=cyan,
}
\usepackage{mathtools}
\usepackage{float}   %this package is for placing graphs and tables in where the TeX is
\graphicspath{{images/}{../images/}}

\usepackage{blindtext}

\usepackage{subfiles}
\usepackage{verbatim}
\providecommand{\EqDir}{Equations}
\providecommand{\RefDir}{References}
\providecommand{\TableDir}{Tables}

\usepackage{apacite}


\title{A Replication of Carroll(1997)}
\author{Yusuf Suha Kulu, Jeongwon (John) Son, and Mingzuo Sun}
\date{}

\begin{document}
\linespread{2}
\maketitle

\section{Summary}

This paper argues that the saving behaviour of a household is better described by the buffer stock version of the Life Cycle/Permanent Income Hypothesis (LC/PIH) than the traditional version of it. Buffer Stock Consumers set average consumption growth equal to average labor income growth, regardless of tastes. The buffer stock model predicts a higher marginal propensity to consume (MPC) out of transitory income, higher effective discount rate for future labor income, and a positive sign for the correlation between saving and expected labor income growth.\\

The finite horizon version of the model presented in the paper explains three emprical puzzles.
\begin{itemize}
\item \textbf{Consumption/income parallel:}Aggregate consumption parallels growth in income over periods of more than a few years.
\item \textbf{Consumption/income divergence:} For individual households, consumption is far from current income. This implies the consumption/income parallel does not arise at the household level.
\item \textbf{Stability of the household age/wealth profile:}The effects of the productivity growth slowdown after 1973 on the age/median-wealth profile and the extraordinarily high volatility of the household liquid wealth are explained. 
\end{itemize}

\newpage 

The Traditional model is the following:\\
Finite Horizon \\
\input{\EqDir/FiniteHorizon.tex}
Infinite Horizon \\
\input{\EqDir/InfiniteHorizon.tex}
The Euler Equation in the buffer stock version of the model is the following:\\
\input{\EqDir/EulerEquation.tex}


\newpage

\begin{center}
\begin{figure} 
\centerline{\includegraphics[width=6in]{Figures/Figure1a.png}}
\label{figure:1}
\caption{Expected Consumption Growth as a Function of Cash on Hand}
\end{figure}
\end{center}


\newpage

\section{Appendix}

%\input{TableDir/table1.tex}

\begin{table}[H]
	\label{table:1}
	\centerline{\scalebox{.5}{\input{Tables/table1.tex}}}
\end{table}

\newpage

\nocite{*}
\bibliographystyle{apacite}
\bibliography{\RefDir/Carroll1997}


\end{document}